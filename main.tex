%%%%%%%%%%%%%%%%%%%%%%%%%%%%%%%%%%%%%%%%%
%  My documentation report
%  Objetive: Explain what I did and how, so someone can continue with the investigation
%
% Important note:
% Chapter heading images should have a 2:1 width:height ratio,
% e.g. 920px width and 460px height.
%
%%%%%%%%%%%%%%%%%%%%%%%%%%%%%%%%%%%%%%%%%


%----------------------------------------------------------------------------------------
%	PACKAGES AND OTHER DOCUMENT CONFIGURATIONS
%----------------------------------------------------------------------------------------

\documentclass[11pt,fleqn]{book} % Default font size and left-justified equations
\usepackage[top=3cm,bottom=3cm,left=3.2cm,right=3.2cm,headsep=10pt,letterpaper]{geometry} % Page margins

\usepackage{xcolor} % Required for specifying colors by name
\definecolor{ocre}{RGB}{52,177,201} % Define the orange color used for highlighting throughout the book

% Font Settings
\usepackage{avant} % Use the Avantgarde font for headings
%\usepackage{times} % Use the Times font for headings
\usepackage{mathptmx} % Use the Adobe Times Roman as the default text font together with math symbols from the Sym­bol, Chancery and Com­puter Modern fonts
\usepackage{microtype} % Slightly tweak font spacing for aesthetics
\usepackage[utf8]{inputenc} % Required for including letters with accents
\usepackage[T1]{fontenc} % Use 8-bit encoding that has 256 glyphs
\usepackage{amsthm}
\usepackage{multirow}

% Bibliography
%\usepackage[style=alphabetic,sorting=nyt,sortcites=true,autopunct=true,babel=hyphen,hyperref=true,abbreviate=false,backref=true,backend=biber]{biblatex}
\usepackage{biblatex}
\addbibresource{bibliography.bib} % BibTeX bibliography file
\defbibheading{bibempty}{}

\input{structure} % Insert the commands.tex file which contains the majority of the structure behind the template

%----------------------------------------------------------------------------------------
%	Definitions of new commands
%----------------------------------------------------------------------------------------

\def\R{\mathbb{R}}
\newcommand{\cvx}{convex}
\begin{document}

%----------------------------------------------------------------------------------------
%	TITLE PAGE
%----------------------------------------------------------------------------------------

\begingroup
\thispagestyle{empty}
\AddToShipoutPicture*{\put(0,0){\includegraphics[scale=1.25]{esahubble}}} % Image background
\centering
\vspace*{5cm}
\par\normalfont\fontsize{35}{35}\sffamily\selectfont
\textbf{IOCC238 SEMESTRE I 2021}\\
{\LARGE Estructuras de Aceros I}\par % Book title
\vspace*{1cm}
{\Huge Apuntes de clases}\par % Author name
\endgroup

%----------------------------------------------------------------------------------------
%	COPYRIGHT PAGE
%----------------------------------------------------------------------------------------

\newpage
~\vfill
\thispagestyle{empty}

\noindent Copyright \copyright\ 2021 Juan Patricio Reyes Cancino\\ % Copyright notice

\noindent \textsc{Universidad Austral de Chile}\\

\noindent \textsc{Instituto de obras civiles}\\

\noindent \textit{Versión original, Marzo 2021} % Printing/edition date

%----------------------------------------------------------------------------------------
%	TABLE OF CONTENTS
%----------------------------------------------------------------------------------------

\chapterimage{head1.png} % Table of contents heading image

\pagestyle{empty} % No headers

\tableofcontents % Print the table of contents itself

%\cleardoublepage % Forces the first chapter to start on an odd page so it's on the right

\pagestyle{fancy} % Print headers again

%----------------------------------------------------------------------------------------
%
%	CAPITULO 1
%
\chapterimage{../Pictures/head2.png} % Chapter heading image
\chapter{Introducción general}

\section{Generalidades}
\section{La estructura en acero y sus componentes}
\section{Comportamiento del acero estructural}
\section{Métodos de diseño}
\section{Normativa técnica}

%
%	CAPITULO 2
%

\chapterimage{../Pictures/head2.png} % Chapter heading image
\chapter{Elementos en tracción}
\section{Tipos de elementos en tracción}
\section{Área neta efectiva}
\section{Diseño de elementos en tracción}
\section{Ejemplos de aplicación}
\section{Ejercicios}
%
%	CAPITULO 3
%

\chapterimage{../Pictures/head2.png} % Chapter heading image
\chapter{Elementos en compresión}
\section{Tipos de elementos en compresión}
\section{Pandeo en columnas}
\begin{example}[Cálculo de la carga crítica de pandeo]
\[\left.\begin{matrix}
P u2-k\left(\frac{6}{L}\cdot u2-\frac{3}{L}\cdot u1\right)=0\\ P u1-k\left(\frac{6}{L}\cdot u1-\frac{3}{L}\cdot u2\right)=0

\end{matrix}\right\}\Rightarrow \begin{matrix}
(P-\frac{6k}{L})\cdot u1+\frac{3k}{L}\cdot u2=0\\
(P-\frac{6k}{L})\cdot u2+\frac{3k}{L}\cdot u1=0
\end{matrix}\\ \]

\[\begin{pmatrix}
P-\frac{6k}{L} &\frac{3k}{L} \\ 
 \frac{3k}{L}&P-\frac{6k}{L} 
\end{pmatrix}\begin{Bmatrix}
u1\\u2

\end{Bmatrix}=\begin{Bmatrix}
0\\0

\end{Bmatrix}\Rightarrow (P-\frac{6k}{L})^2 u2+\frac{3k}{L}^2=0\]
\[
P_1=\frac{9k}{L}\\
P_2=\frac{3k}{L}\]
\begin{align*}
M_r=M_e\\
M_r=k\cdot\alpha\\
M_e=P\cdot u\\
P\cdot u1-k(\alpha_1-\alpha_2)=0\\
P\cdot u2-k(\alpha_3-\alpha_2)=0\\
u1=\alpha_1\frac{L}{3}\rightarrow \alpha_1=\frac{3}{L}u1\\
u2=\alpha_3\frac{L}{3}\rightarrow \alpha_3=\frac{3}{L}u2\\
u2-u1=\alpha_2\frac{L}{3}\rightarrow \alpha_2=\frac{3}{L}(u2-u1)\\
\frac{5}{12}L(M_1\cdot M_2)L=h\\
M_1=P\cdot \Delta\\
\end{align*}
\end{example}
\begin{exercise}[Ejercicio de aplicación pandeo de columna]
\begin{align*}
\frac{5}{12}(M_i\cdot M_k)L&=EI\Delta\\
\frac{1}{3}L(M_i\cdot M_k)&=EI\Delta\\
M_i&=P\cdot \Delta\\
M_k&=1\cdot h\\
\frac{5}{12}(P\cdot \Delta)\cdot(1\cdot h)\cdot h \cdot\frac{1}{EI_s}+\frac{1}{3}(P\cdot \Delta)\cdot(1\cdot h)\cdot L \cdot\frac{1}{EI_b}&=\Delta\\
\frac{5}{12}(P\cdot \Delta)\cdot h^2 +
\frac{1}{3}(P\cdot \Delta)\cdot h\cdot L \cdot\frac{I_s}{I_b}&=EI_s\Delta\\
P\cdot h(\frac{5}{12}\cdot h +
\frac{1}{3}\cdot L \cdot\frac{I_s}{I_b})&=EI_s\\
P_{cr}&=\frac{EI_s}{h(\frac{5}{12}\cdot h +
\frac{1}{3}\cdot L \cdot\frac{I_s}{I_b})}\\
I_b&=2I_s\\
h&=2L\\
P_{cr}&=\frac{2EI_s}{h^2}\\
P_{cr}&=\frac{\pi^2 EI_s}{\beta^2 h^2}\\
\beta&=\frac{\pi}{\sqrt{2}}=2,22
\end{align*}
\end{exercise}
\section{Diseño de elementos en compresión}
\section{Ejemplos de aplicación}
\\
\begin{example}[Ejemplo de diseño a compresión de marco simple.]
Se deberá diseñar el perfil de la estructura de la figura 3, considerando los siguientes parámetros:
Acero $A36$ $(253 MPa)$\\
Elasticidad $E=200.000 MPa$\\
$H=4,5 m$\\
$L=6,0 m$\\
\\
Las cargas en los extremos serán las siguientes:\\
$DL=100 tonf$\\
$LL=125 tonf$\\
\\
Con combinaciones de carga para ASD y LRFD:\\
ASD:\\
$C=DL+LL$\\
LRFD:\\
$C=1,2DL+1,6LL$\\
\vspace{5mm}
Considerar un perfil W12x120\\
\vspace{5mm}
\textit{Solución:}\\
\vspace{5mm}
Primero se deberán calcular las combinaciones de carga para ASD y LRFD:\\
ASD\\
$C_{ASD}=1000kN+1250kN=2250 kN$\\
$P_{t}=2250kN$ Carga de trabajo
LRFD:\\
$C_{LRFD}=1,2\cdot100kN+1,6\cdot 125kN=3200 kN $\\
$P_{n}=3200kN$ Carga nominal
% Please add the following required packages to your document preamble:
% \usepackage{multirow}
% Please add the following required packages to your document preamble:
% \usepackage{multirow}
\begin{table}[h!]
\centering
\begin{tabular}{ccccccc}
\hline
\multirow{2}{*}{W12x120} & b_f      & t_f      & A        & d       & t_w     & r     \\ \cline{2-7} 
                         & 12.32in & 1.105in & 35.3in^2 & 13.12in & 0.71in & 5.5in \\ \hline
\end{tabular}
\end{table}
Obtenidos los parámetros geométricos del perfil, se deberá comenzar con la comprobación, en un primer acercamiento se procederá a calcular las esbelteces del alma y el ala mediante el uso de las relaciones ancho-espesor:\\
Esbeltez del ala:\\
\begin{align*}
\frac{b_{f}}{2t_f}<0,56\sqrt{\frac{E}{F_y}}
\end{align*}
De acuerdo con los parámetros geométricos del perfil y del material se tiene que:
\begin{align*}
    \frac{12,32in}{2\cdot 1,105in}=5,58\\
    0,56\sqrt{\frac{200.000MPa}{253MPa}}=15,74\\
    5,58<15,74
\end{align*}
Por lo tanto no hay pandeo local del ala.
También deberá ser necesario realizar las verificaciones de esbeltez del alma, por lo que se procederá a comprobar las relaciones ancho espesor como sigue:
\begin{align*}
    \frac{d}{t_w}<1,49\sqrt{\frac{E}{F_y}}
\end{align*}
De acuerdo a lo indicado en el párrafo anterior, se tiene que para el alma del perfil de acero:
\begin{gather*}
    \frac{13,12in}{0,79in}=18,48\\
    1,49\sqrt{\frac{200.000MPa}{253MPa}}=41,89\\
    18,48<41,89
\end{gather*}
Tampoco hay presencia de pandeo local del alma. De hecho en ambos casos los valores son menos de un 45\% del valor límite.
Como ambos casos de las relaciones ancho espesor resultan menores que los limites, se dice que la sección es \textbf{compacta} y por lo tanto se deberán usar las formulaciones para ello.\\
Para poder resolver el problema de diseño, se necesita evaluar la esbeltez del perfil, es decir la relación $\frac{kL}{r}$ y para ello es necesario saber el valor de $k$ que es el factor de longitud efectiva de los elementos sometidos a cargas de compresión. Por lo que se es necesario realizar el cálculo de este factor.\\
Se utiliza el método de doble integración, en donde se tiene que:\\
\begin{align*}
    EI\delta&=\frac{5}{12}P\cdto\delta\cdot h\cdot 0,5h\cdot 2+P\cdot\delta\cdot 0,5h\cdot 0,66h\frac{1}{3}\cdot2\\
    P&=\frac{EI}{0,639h^2}
\end{align*}
Como el $P_{cr}$ Euler es igual a:
\begin{align*}
    P_{cr} &= \frac{\pi^2 EI}{(kh)^2}\\
    k &= \sqrt{0,639\cdot\pi^2}\\
    k &= 2,51
\end{align*}
Luego, la esbeltez del sistema estaría dada como:
\begin{align*}
    \frac{kL}{r}&=\frac{2,51\cdot 4,5cm}{14cm}
    &=80,67
\end{align*}
De acuerdo con la ecuación E3 del AISC, este valor se debe comparar con:
\begin{align*}
    4,71\sqrt{\frac{E}{F_y}}&=132,43\\
    80,67&<132,43
\end{align*}
Como $\frac{kL}{r}$ es menor que $4,71\sqrt{\frac{E}{F_y}}$, se deberá ocupar la fórmula E3-2 del AISC, es decir:
\begin{align*}
    F_{cr}&=\left[0,658^\frac{F_y}{F_e}\right]\cdot F_y\\
    Fe&=\frac{\pi^2 E}{\frac{kL}{r}} \text{\hspace{15mm}(Tensión crítica de Euler)}
\end{align*}
Luego, reemplazando las ecuaciones anteriores con los parámetros anteriormente calculados, se tiene:
\begin{align*}
    F_e&=303,26 MPa
    F_{cr}&=178,43 MPa
\end{align*}
Finalmente, las tensiones admisibles ASD y LRFD respectivamente son:\\
Para ASD:\\
 \begin{align*}
     F_{adm}&=\frac{F_{cr}}{\Omega}\\
     &=\frac{178,43 MPa}{1,67}\\
     &=106,85 MPa\\
     A&=228cm^2\\
     P_{adm}&=2436kN\\
     P_{t}&<P_{adm}\hspace{10mm} \rightarrow\textbf{\hspace{5mm}Cumple.}
 \end{align*}
Para LRFD:\\
 \begin{align*}
     F_{n}&=\phi F_{cr}\\
     &=0,9\cdot 178,43 MPa\\
     &=160,59 MPa\\
     A&=228cm^2\\
     P_{n}&=3657kN\\
     P_u&<P_{n}\hspace{10mm} \rightarrow\textbf{\hspace{5mm}Cumple.}
 \end{align*}
\end{example}
\section{Ejercicios}
%
%	CAPITULO 4
%

\chapterimage{../Pictures/head2.png} % Chapter heading image
\chapter{Elementos en flexión}
\section{Análisis de elementos en flexión}
Se comenzará a estudiar una viga simplemente apoyada de sección rectangular sometida a una carga uniformemente distribuida.\\
Se supone que en principio que la viga solo incursiona en el rango lineal elástico del material, por lo que no existe plastificación.\\
Esto significa que solo se considera el módulo elástico del perfil.
El material será acero para efectos de análisis.
\begin{align}
    F_y &= \frac{M\cdot \Bar{y}}{I}\\
    F_y &= \frac{M}{S}
\end{align}
donde:
\begin{align}
    S = \frac{I}{\Bar{y}} \hspace{10mm}\text{ó}\hspace{10mm} W = \frac{I}{\Bar{y}}
\end{align}
Se tiene entonces de acuerdo a las consideraciones geométricas y propiedades mecánicas del material que:\\
\begin{align} \label{eq_epsilon}
    \epsilon &= \frac{y}{\rho}\\
    M &= \int_A yfdA\\
\end{align}
como $f=E\cdot\epsilon$, entonces la ecuación anterior queda:\\
\begin{align}
    M &= \frac{E}{\rho}\int_A y^2dA=\frac{EI}{\rho}
\end{align}
Si se reemplaza $\rho$ de acuerdo con la ecuación (\ref{eq_epsilon}), se tiene que:\\
\begin{align}
    M &= \frac{EI\epsilon}{y} = \frac{fI}{y}
\end{align}
En una viga rectangular, el momento $M_y$ elástico máximo, está dado por la relación entre la tensión de fluencia $F_y$ y la inercia dividida por la distancia desde el eje neutro a la fibra extrema.\\
Esto es:
\begin{align}
    M_y &= \frac{F_y\cdot I}{\Bar{y}}
\end{align}
La inercia de una sección rectangular de ancho $b$ y alto $d$:
\begin{align}
    I &= \frac{bd^3}{12}
\end{align}
En este caso $\Bar{y}=d/2$ y la tensión $f = F_y $ por lo que la ecuación del momento resultante sería:
\begin{align}
    M_y &= \frac{F_y\cdot bd^3/12}{d/2}\\
        &= \frac{F_ybd^2}{6}
\end{align}
Lo que es precisamente el resultado del momento elástico para una viga de sección rectangular.\\
\begin{example}[Viga W sometida a flexión pura]
Se le solicita calcular el módulo elástico, plástico y el factor de forma de una viga W sometida a flexión, la sección sería un perfil W18x50.\\
\\
\textit{Solución:}\\
Primero se deben conocer las propiedades geométricas de la sección:\\
\begin{table}[h!]
\centering
\begin{tabular}{ccccccc}
\hline
\multirow{2}{*}{W18x50} & $b$      & $t_f$      & $d$        & $t_w$       & $I_y$     & $S_y$     \\ \cline{2-7} 
                         & 190mm & 14,5mm & 457mm & 9,0mm & 33.415$cm^4$ & $1462cm^3$ \\ \hline
\end{tabular}
\end{table}
La módulo elástico se puede calcula directamente de las propiedades, es decir, es la inercia dividida por la mitad de la altura. Lo que se puede corroborar mediante el análisis del momento elástico:
\begin{align*}
    M_y &= \frac{F_y}{6d}(bd^3-(b-t_w)(d-2t_f)^3)\\
        &= F_y\cdot S_y\\
    S_y &= \frac{bd^3-(b-t_w)(d-2t_f)^3}{6d}\\
        &=1438cm^3 (\approx 1462cm^3)\\
\end{align*}
El módulo plástico también es posible de calcular siguiendo las indicaciones anteriores:\\
\begin{align*}
    M_p &= \F_y\left(\frac{t_w(d-2t_f)^2}{4}+bt_f\left(\frac{2d-2t_f}{2}\right)\right)\\
        &= F_y\cdot Z\\
      Z &= \frac{t_w(d-2t_f)^2}{4}+bt_f\left(\frac{2d-2t_f}{2}\right)\\
        &= 1631cm^3\\
\end{align*}
Luego el factor de forma resulta:\\
\begin{align*}
    \frac{M_p}{M_y} &= \frac{Z}{S_y}\\
                    &= \frac{1438cm^3}{1631cm^3}\\
                    &= 1,134 (\approx 1,14)\\
\end{align*}
\end{example}
\section{Pandeo lateral torsional}
En general, las secciones que se usan como vigas donde el momento de inercia en torno al eje fuerte es considerable mayor que el del el eje débil. Esto es así para producir secciones optimizadas y económicas.\\
En consecuencia, son secciones débiles a torsión y a flexión en torno a su eje débil, por lo que en la algunos casos se deberá disponer de un arriostramiento para ser sostenidas ya que es posible que se produzca un fenómeno conocido como pandeo lateral o pandeo lateral torsional, que es a la vez un alabeo lateral y una torsión.
\begin{align}
    \label{Eq_Mx}
    -EI_x\frac{d^2v}{dz^2} &= M_x\\
    \label{Eq_Mb}
    -EI_y\frac{d^2u}{dz^2} &= M_x\beta\\
\end{align}
El momento torsor por su lado está dado por la ecuación \ref{Eq_T}, que se encuentra derivada en el capítulo 3:
\begin{align}
    \label{Eq_T}
    GJ\frac{d\beta}{dz}-EC_w\frac{d^3\beta}{dz^3} &= T
\end{align}
Donde T sería el momento torsor y según el diagrama estaría dado por:
\begin{align}
    \label{Eq_T_Mx}
    T &= M_x\frac{du}{dz}
\end{align}
Por lo tanto, la ecuación \ref{Eq_T} resulta en:
\begin{align}
    GJ\frac{d\beta}{dz}-EC_w\frac{d^3\beta}{dz^3} - M_x\frac{du}{dz}&= 0
\end{align}
Esta ecuación se puede reducir diferenciándola con respecto a $z$ y eliminando $d^2u/dz^2$ utilizando la ecuación \ref{Eq_Mb}:\\
\begin{align}
\label{Eq_PLT}
     EC_w\frac{d^4\beta}{dz^4}- GJ\frac{d^2\beta}{dz^2} - \frac{M_x^2}{EI_y}\beta &= 0
\end{align}
Las condiciones de contorno para la solución de la ecuación \ref{Eq_PLT} son las siguientes:
\begin{align*}
    \beta_{L=0} &= 0\\
    \beta_{L=L} &= 0\\
    \beta_{L=L/2} &= \beta_{L/2}
\end{align*}
Por lo tanto la ecuación \ref{Eq_PLT} se satisface si:
\begin{align}
    \beta &= \beta_{L/2}\sin{\frac{\pi z}{L}}
\end{align}
Cuyas derivadas son las siguientes:
\begin{align}
    \frac{d^2\beta}{dz^2} = \frac{\pi^2}{L^2}\beta_{L/2}\sin{\frac{\pi z}{L}}\\
    \frac{d^4\beta}{dz^4} = \frac{\pi^4}{L^4}\beta_{L/2}\sin{\frac{\pi z}{L}}
\end{align}
Por lo que la ecuación \ref{Eq_PLT} resulta finalmente:
\begin{align}
    \left(EC_w\frac{\pi^4}{L^4}+ GJ\frac{\pi^2}{L^2} - \frac{M_x^2}{EI_y}\right)\beta_{L/2}\sin{\frac{\pi z}{L}} &= 0
\end{align}
Lo que se satisface si $\beta_{L/2}=0$, lo que sería la solución trivial, es decir, no existiendo casi torsión, o cuando:
\begin{align}
\label{Eq_TLP2}
      EC_w\frac{\pi^4}{L^4}+ GJ\frac{\pi^2}{L^2} &= \frac{M_x^2}{EI_y}   
\end{align}
La ecuación \ref{Eq_TLP2} es válida solo para las condiciones de contorno expuestas anteriormente. Para considerar otras condiciones de apoyo, en donde por ejemplo pueda existir torsión en el inicio del tramo, una formulación más general sería:
\begin{align}
\label{Eq_TLP3}
    M_{xcr}^2 &= C_b^2\left[\frac{\pi^2}{(kL)^2}EI_yGJ + \frac{\pi^4}{(kL)^4}EI_y\cdot EC_w\right]
\end{align}
Donde:\\
$C_b$ : Coeficiente de momento.\\
$k$ : Coeficiente de largo efectivo.\\
$C_w$ : Constante de alabeo.\\
$J$ : Constante torsional.\\
El coeficiente de momento $C_b$ puede ser calculado utilizando la siguiente relación:
\begin{align}
    C_b &= \frac{12.5M_{max}}{2.5M_{max}+3M_A+4M_B+3M_C}
\end{align}
Donde:\\
$M_{max}$ : Es el mayor momento absoluto en el tramo no arriostrado.\\
\hspace{10mm}$M_A$ : Es el valor absoluto del momento a 1/4 del tramo.\\
\hspace{10mm}$M_B$ : Es el mayor valor absoluto del momento a 1/2 del tramo.\\
\hspace{10mm}$M_C$ : Es el mayor absoluto del momento a 2/4 del tramo no arriostrado.\\

Para el rango inelástico, la ecuación \ref{Eq_TLP3} puede generalizarse utilizando una factor de modificación $\tau$:
\begin{align}
    M_{xcrIn}^2 &= M_{xcr}^2\cdot\tau\\
    \tau &= \frac{E_t}{E}
\end{align}
$E_t$ : Módulo tangente.
\section{Pandeo local de elementos de una viga}
Para el análisis del pandeo local de alas y almas de la sección transversal de vigas de acero sometidas a flexión, se realiza con el estudio de estabilidad de placas delgadas. Es decir se busca la tensión crítica de pandeo, para placas cargadas en los bordes de su plano.\\
Para esto se deberá solucionar la ecuación diferencial de equilibrio de placas planas delgadas:
\section{Tensiones de diseño}
\subsection{Diseño por flexión ASD}
\subsection{Diseño por cortante LRFD}
\section{Ejemplos de aplicación}
\section{Ejercicios}
%
%	CAPITULO 5
%

\chapterimage{../Pictures/head2.png} % Chapter heading image
\chapter{Elementos en flexión compuesta}
\section{Tipos de elementos en flexión compuesta}
\section{Diseño de elementos en flexión compuesta}
\section{Ejemplos de aplicación}
\section{Ejercicios}
%
%	CAPITULO 6
%

\chapterimage{../Pictures/head2.png} % Chapter heading image
\chapter{Elementos en Torsión}
\section{Tipos de elementos en torsión}
\section{Diseño de elementos en torsión}
\section{Ejemplos de aplicación}
\section{Ejercicios}
%
%	CAPITULO 7
%

\chapterimage{../Pictures/head2.png} % Chapter heading image
\chapter{Uniones estructurales}
\section{Tipos de uniones}
\section{Diseño de uniones soldadas}
\section{Diseño de uniones apernadas}
\section{Ejemplos de aplicación}
\section{Ejercicios}
%
%	CAPITULO 8
%

\chapterimage{../Pictures/head2.png} % Chapter heading image
\chapter{El proyecto estructural en acero}
\section{Análisis de estructuras de acero}
\section{Diseño de estructuras de acero}
\section{Taller de diseño en acero}
\section{Palabras finales}


%
%	CHAPTER 1
%

\chapterimage{head2.png} % Chapter heading image
\chapter{Convex Sets}
\section{Convexity}
\subsection{Cone}
\begin{definition}[Cone]
A set $K \in \R^n$, when $x \in K $ implies $\alpha x \in K$.
\end{definition}
A non convex cone can be hyper-plane.\\
For convex cone $x + y \in K, \forall x,y \in K$.\\
Cone don't need to be "pointed". e.g. \\
Direct sums of cones $C_1 + C_2 = \{ x = x_1+x_2 | x_1 \in C_1, x_2 \in C_2 \}$.\\
\begin{example}
$S_1^n  \{ X | X=X^n ,\lambda(x) \ge 0\}$\\
A matrix with positive eigenvalues.
\end{example}

\subsubsection{Operations preserving convexity}
\begin{itemize}
\item[Intersection] $C  \cap_{i \in \mathbb{I}}C_i$
\item[Linear map] Let $A : \mathbb{R}^n \to  \R^n$ be a linear map. If $C \in \R^n$ is convex, so is $A(C) = \{Ax \forall x \in C \}$
\item[Inverse image] $A^{-1}(D) = \{ x \in \R |Ax \in D \}$
\end{itemize}

\subsubsection{Operations that induce convexity}
Convex hull on $S = \cap \{C | S\in C, C is convex\}$\\
\begin{example}
$Co \{ x_1,x_2,\cdots,x_m\} = \{ \sum_{i=1}^m \alpha_i x_i | \alpha \in \delta_m \}$
\end{example}
For a convex set $x \in C \Rightarrow x = \sum \alpha_i x_i$. 
\begin{theorem}[Carathéodory's theorem]
If a point $x \in \R^d$ lies in the convex hull of a set $P$, there is a subset $P'$ of $P$ consisting of $d + 1$ or fewer points such that $x$ lies in the convex hull of $P'$. Equivalently, x lies in an r-simplex with vertices in P.
\end{theorem}

\section{Convex Functions}
\begin{definition}[Convex function]
Let $C \in \R^n$ be convex, $f:C \to \R$ is convex on f if $x,y \in C \times C$. $\forall \alpha \in (0,1)$, $f(\alpha x + (1-\alpha) y) \le f(\alpha x) + f((1-\alpha) y)$
\end{definition}

\begin{definition}[Strictly Convex function]
Let $C \in \R^n$ be convex, $f:C \to \R$ is strictly convex on f if $x,y \in C \times C$. $\forall \alpha \in (0,1)$, $f(\alpha x + (1-\alpha) y) \langle f(\alpha x) + f((1-\alpha) y)$
\end{definition}

\begin{definition}[Strongly convex]
$f:C \to \R$ is strongly convex with modules $u \ge 0$ if $f - \frac{1}{2}u || \cdot ||^2$ is convex.
\end{definition}
Interpretation: There is a convex quadratic $\frac{1}{2}u || \cdot ||^2$ that lower bounds f.
\begin{example}
$\min_{x \in C} f(x) \leftrightarrow \min \bar{f}(x)$
Useful to turn this into an unconstrained problem. \\
$$\bar{f}(x) = \begin{cases}
f(x) \quad if x \in C \\
\infty \quad  elsewhere
\end{cases}$$
\end{example}
\begin{definition}
A function $f : \R^n \to \R \cup \infty \ \bar{\R}$ is convex if $x,y \in \R^n \times \R^n$, $\forall x,y , \bar{f}(\alpha x + (1-\alpha) y) \le f(\alpha x) + f((1-\alpha) y)$
\end{definition}
Definition 1 is equivalent to definition 2 if $f(x) = \infty$.
\begin{example}
$f(x) = \sup_{j \in J} f_j(x)$
\end{example}

\subsection{Epigraph} 
\begin{definition}[Epigraph]
For $f: \R^n \rightarrow \bar{R}$, its epigraph $epi(f) \in \R^{n+1} is the set epi(f) \{ (x,\alpha) | f(x) \in \alpha \}$
\end{definition}
Next: a function is convex i.f.f. its epigraph is convex \cite{article_key}.

\begin{definition}
A function $f : C \rightarrow \R, C \in \R^n$ is convex if $\forall x, y \in C$, $f(ax + (1-a)x) \le af(x) + (1-a)f(x) \quad \forall a \in (0,1)$.\\ 
Strict convex: $x \neq y \Rightarrow f(ax + (1-a)x) \le af(x) + (1-a)f(x) $
\end{definition}
\begin{remark}
$f$ is convex $\Rightarrow$ $-f$ is concave \cite{book_key}.
\end{remark}
Level set: $S_{\alpha}f = \{ x | f(x) \le \alpha \}$.\\ 
$S_{\alpha}f$ is convex $\Leftrightarrow$ $f$ is convex. \\
\begin{definition}[Strongly convex]
$f : C \rightarrow \R$ is strongly convex with modules $\mu$ if $\forall x, y \in C$, $\forall \alpha \in (0,1)$, $f(ax + (1-a)x) \le af(x) + (1-a)f(x) - \frac{1}{2\mu}\alpha(1- \alpha) \|x-y\|^2$.
\end{definition}

\begin{remark}
\begin{itemize}
\item $f$ is 2nd-differentiable, $f$ ix \cvx $\iff$ $\nabla^2f(x) \rangle  0$.
\item $f$ is strongly \cvx $\iff$ $\nabla^2f(x) \rangle  \mu I$ $\iff$ $x \ge \mu$
\end{itemize}
\end{remark}
\begin{definition}[2]
$f : \R^n \to \bar{\R} $ is \cvx  if $x, y  \in \R , \alpha \in (0,1), f(ax + (1-a)x) \le af(x) + (1-a)f(x)$.  
\end{definition}
The effective domain of $f$ is $dom f = \{x | f(x) \langle + \infty \}$ 
\begin{example}[ludcator function]
$\delta_c(x) = \begin{cases}
0 \quad  x \in C \\
+ \infty \quad elsewhere
\end{cases}$.\\
$dom \space \delta_c(x) = C$
\end{example}
\begin{definition}[Epigraph]
The epigraph of f is $epi \space f = \{(x,\alpha) | f(x) \le \alpha\}$
\end{definition}
The graph of $epi \space f$ is $\{ (x,f(x) | x \in dom \space f\}$.
\begin{definition}[III]
A function $f : \R^n \to \bar{\R}$ is %\cvx  if $\epi \space f $ is \cvx
\end{definition}
\begin{theorem}
$f : \R^n \to \bar{\R}$ is \cvx  $\iff$ $\forall x,y \in \R^n, \alpha \in (0,1), f(ax + (1-a)x) \le af(x) + (1-a)f(x)$.
\end{theorem}
\begin{proof}
$\Rightarrow$ take $x,y \in dom \space f$, $(x,f(x)) \in epi \space f$,$(y,f(y)) \in epi \space f$.
\end{proof}

\begin{example}[Distance]
Distance to a \cvx  set $d_c(x) = \inf \{ \| z-x \| | z \in C \}$. Take any two sequence $\{ y_k\} and \{ \bar{y}_k\} \subset C$ s.t. $\| y_k - x\| \to d_c(x)$, $\| \bar{y}_k - \bar{x}\| \to d_c(\bar{x})$. $z_k = \alpha y_k + (1 - \alpha) \bar{y}_k$.
\begin{align*}
d_c(\alpha x + (1-\alpha) \bar{x}) &\le \| z_k - \alpha x - (1 - \alpha) \bar{x}\| \\
& = \| \alpha(y_k - x) + (1 - \alpha)(\bar{y}_k - \bar{x})\| \\
& \le \alpha \| y_k - x\| + (1 - \alpha ) \|\bar{y}_k - \bar{x}\|
\end{align*}
Take $k \to \infty$, $d_c(\alpha x + (1 - \alpha) \bar{x}) \le \alpha d(x) + (1 - \alpha) d(\bar{x})$
\end{example}
\begin{example}[Eigenvalues]
Let $X \in S^n := \{ n \times n symmetric matrix\}$. $\lambda_1(x) \ge \lambda_2(X) \ge \ldots \ge \lambda_n(x)$.\\
$f_k(x) = \sum_{1}^n \lambda_i(x)$.\\
Equivalent characterization 

\begin{align*}
f_k(x) & = \max\{ \sum_{i} v_i^T Xv_i | v_i \perp v_j , i \neq j\} \\
& =  \max\{ tr( V^TXV | V^T V = I_k \} \\
\max \{tr(VV^TX) \} \text{by circularity}
\end{align*}
Note $\langle A,B\rangle  = tr(A,B)$ is true for symmetric matrix. \\
$\langle A,A\rangle  = |A |_F^2 = \sum_{i} A_{ii}^2$
\end{example}

\section{Support Function}
Take a set $C \in \R^n$, not necessarily convex.The support function is $\sigma_C = \R^n \to \bar{\R}$. $\sigma_C(x) = \sum \{ \langle x,u\rangle  | u \in C\}$.
\includegraphics[scale=0.5]{1_1.png}
\begin{fact}
The support function binds the supporting hyper-plane.
\end{fact}

Supporting functions are
\begin{itemize}
\item Positively homogeneous\\
$\sigma_C(\alpha x) = \alpha \sigma_C(x) \forall \alpha \rangle  0$ \\
$\sigma_C(\alpha x ) = \sup_{u \in C} \langle \alpha x, u\rangle  = \alpha \sup_{u \in C} \langle x, u\rangle  = \alpha \sigma_C(x)$
\item Sub-linear( a special case of convex, linear combination holds $\forall \alpha$.\\
$\sigma_C(\alpha x + (1 - \alpha) y ) = \sup_{u \in C} \langle \alpha x + (1 - \alpha) y,u\rangle  \le \alpha\sup_{u \in C}\langle x,u\rangle  + (1 - \alpha)\sup_{u \in C}\langle y,u\rangle  $
\end{itemize}
\begin{example}[L2-norm]
$\| x \| = \sup_{u \in C} \{ \langle x, u \rangle, u \in \R^n \}$.\\
$\|x \|_p = \sup \{ \langle x, u \rangle, u \in B_q \}$ where $\frac{1}{p} + \frac{1}{q} = 1$. $B_q = \{ \|x \|_q \le 1\}$.\\
The norm is 
\begin{itemize}
\item Positive homogeneous
\item sub-linear
\item If $0 \in C$, $\sigma_C$ is non-negative.
\item If $C$ is central-symmetric, $\sigma_C(0) = 0$ and $\sigma_C(x) = \sigma_C(-x)$
\end{itemize}
\end{example}

\begin{fact}[Epigraph of a support function]
$epi \space \sigma_C = \{ (x,t) | \sigma_C(x) \le t\}$.
Suppose $(x,t) \in epi \space \sigma_C$. Take any  $\alpha > 0$. $\alpha(x,t) = (\alpha x, \alpha t)$.\\
$\alpha \sigma_C(x) = \alpha \sigma_C(x) \le \alpha t$. $\alpha(x,c) \in epi 
\sigma_C$\\
\includegraphics[]{1_2}
\end{fact}

\section{Operations Preserve Convexity of Functions}
\begin{itemize}
\item Positive affine transformation \\
$f_1,f_2,\ldots,f_k \in \space cvx \R^n$.\\
$f = \alpha_1 f_1 + \alpha_2 f_2 + \ldots + \alpha_k f_k$
\item Supremum of functions. Let $\{ f_i \}_{i \in I}$ be arbitrary family of functions. If $\exists x \sup_{j \in J} f_j(x) < \infty \Leftrightarrow f(x) = \sup_{j \in J} f_j(x) $\\
\includegraphics[]{1_3}
\item Composition with linear map.\\
$f \in cvx \R^n$, $A:\R^n \to \R^m$ is a linear map.
$f \circ A (x) = f(Ax) \in cvx \R^n$\\
\begin{align*}
f \circ A (x) & = f(A(\alpha x + (1-\alpha) y)) \\
& = f(A \alpha x + (1-\alpha) A y) \\
& \le \alpha f(Ax) + (a - \alpha) f(Ay)
\end{align*}
\end{itemize}

\begin{exercise}[L2-norm]
$\| x \| = \sup_{u \in C} \{ \langle x, u \rangle, u \in \R^n \}$.\\
$\|x \|_p = \sup \{ \langle x, u \rangle, u \in B_q \}$ where $\frac{1}{p} + \frac{1}{q} = 1$. $B_q = \{ \|x \|_q \le 1\}$.\\
The norm is 
\begin{itemize}
\item Positive homogeneous
\item sub-linear
\item If $0 \in C$, $\sigma_C$ is non-negative.
\item If $C$ is central-symmetric, $\sigma_C(0) = 0$ and $\sigma_C(x) = \sigma_C(-x)$
\end{itemize}
\end{exercise}

\printbibliography[title={Referencias}]
\end{document}